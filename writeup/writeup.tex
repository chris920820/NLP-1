\documentclass[a4paper, 11pt]{article}
\setlength{\parindent}{0pt}
\usepackage{latexsym}
\usepackage{amssymb}
\usepackage{times}
%\usepackage[in]{fullpage}
\usepackage{amsmath,amsfonts,amsthm}
\usepackage{graphicx}
\newtheorem{theorem}{Theorem}
\usepackage[utf8]{inputenc}

% Default fixed font does not support bold face
\DeclareFixedFont{\ttb}{T1}{txtt}{bx}{n}{12} % for bold
\DeclareFixedFont{\ttm}{T1}{txtt}{m}{n}{12}  % for normal

% Custom colors
\usepackage{color}
\definecolor{deepblue}{rgb}{0,0,0.5}
\definecolor{deepred}{rgb}{0.6,0,0}
\definecolor{deepgreen}{rgb}{0,0.5,0}

\usepackage{listings}

% Python style for highlighting
\newcommand\pythonstyle{\lstset{
language=Python,
basicstyle=\ttm,
otherkeywords={self},             % Add keywords here
keywordstyle=\ttb\color{deepblue},
emph={MyClass,__init__},          % Custom highlighting
emphstyle=\ttb\color{deepred},    % Custom highlighting style
stringstyle=\color{deepgreen},
frame=tb,                         % Any extra options here
showstringspaces=false            % 
}}


% Python environment
\lstnewenvironment{python}[1][]
{
\pythonstyle
\lstset{#1}
}
{}

% Python for external files
\newcommand\pythonexternal[2][]{{
\pythonstyle
\lstinputlisting[#1]{#2}}}

% Python for inline
\newcommand\pythoninline[1]{{\pythonstyle\lstinline!#1!}}

\begin{document}

\date{\today}
\author{Meng Zhang, Zhouyang Zhang, Chenxi Liu }
\title{\textbf{EN 600.465 \\Natural Language Processing\\Assignment 1\\ }}

\maketitle
\section{Problem 2}
(a) Why does your program generate so many long sentences? Specifically, what grammar rule is responsible and why? What is special about this rule?\\
Answer: \\
(b) The grammar allows multiple adjective, as in \textit{the fine perplexed pickle}. Why do your program's sentences do this so rarely? \\
Answer: \\
(d) Which  numbers  must  you  modify  to  fix  the  problems  in  (a)  and  (b),  making  the  sentences shorter and the adjectives more frequent? (Check your answer by running your generator!) \\
Answer: \\
(e) What other numeric adjustments can you make to the grammar in order to favor more natural sets of sentences?  Experiment.  Hand in your grammar file in a file named grammar2, with comments that motivate your changes, together with 10 sentences generated by the grammar. \\
Answer: \\




%%%%%%%%%%%%%%%%%%%%%%%%%%%%%%%%%%

\end{document}



%%%%%%%%%%%%%%%%%%%%%%%%%%%